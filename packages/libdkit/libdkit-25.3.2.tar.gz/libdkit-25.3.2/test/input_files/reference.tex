\section{heading with \textbf{bold}}
This is a first paragraph testing the \emph{viability} of
the md to latex conversion.
\subsection{Heading L2}
text 2 
\subsubsection{Heading L3}
text 3
\section{Verbatim}
\subsection{inline}
This is \lstinline!inline! verbatim text
\subsection{simple}

\begin{verbatim}
line 1 of verbatim text
line 2 of verbatim texta
\end{verbatim}
\subsection{code}
The following represent verbatim code in python:


\begin{lstlisting}[language=python]
import string
print(string.punctiuation)
\end{lstlisting}
\subsection{latex}
This is how to include verbatim latex:

\emph{verbatim text}
\subsection{jsoninclude}
This is how to add json directly to the ast:

This is the included data
\subsection{verbatim}

\begin{verbatim}
verbatim
\end{verbatim}
\section{Lists}
\subsection{Unorderedlists}
text in the \emph{sub} heading:

\begin{itemize}
\item \item \textbf{one}

\item \item two

\end{itemize}

Final
\subsection{Ordered lists}

\begin{enumerate}
\item One
\item Two
\end{enumerate}

\section{Image}
Here is an image 
\begin{center}
\includegraphics{test}
\end{center}
.
\section{Math}
Simple $m$ symbol
\section{Table}
A simple table:
\section{An h1 header}
==============
Paragraphs are separated by a blank line.
2nd paragraph. \emph{Italic}, \textbf{bold}, \lstinline!monospace!. Itemized lists
look like:

\begin{itemize}
\item \item this one

\item \item that one

\item \item the other one

\end{itemize}

Note that --- not considering the asterisk --- the actual text
content starts at 4-columns in.

\begin{displayquote}
Block quotes are
written like so.
They can span multiple paragraphs,
if you like.

\end{displayquote}

Use 3 dashes for an em-dash. Use 2 dashes for ranges (ex. "it's all in
chapters 12--14"). Three dots ... will be converted to an ellipsis.
\subsection{An h2 header}
Here's a numbered list:

\begin{enumerate}
\item \item first item

\item \item second item

\item \item third item

\end{enumerate}

Note again how the actual text starts at 4 columns in (4 characters
from the left side). Here's a code sample:

\begin{verbatim}
# Let me re-iterate ...
for i in 1 .. 10 { do-something(i) }
\end{verbatim}
As you probably guessed, indented 4 spaces. By the way, instead of
indenting the block, you can use delimited blocks, if you like:

\begin{verbatim}
define foobar() {
    print "Welcome to flavor country!";
}
\end{verbatim}
(which makes copying & pasting easier). You can optionally mark the
delimited block for Pandoc to syntax highlight it:


\begin{lstlisting}[language=python]
import time
# Quick, count to ten!
for i in range(10):
    # (but not *too* quick)
    time.sleep(0.5)
    print i
\end{lstlisting}
\subsubsection{An h3 header}
Now a nested list:

\begin{enumerate}
\item \item First, get these ingredients:
\linebreak[1]

\begin{itemize}
\item \item carrots

\item \item celery

\item \item lentils

\end{itemize}



\item \item Boil some water.


\item \item Dump everything in the pot and follow
this algorithm:
\linebreak[1]

\begin{verbatim}
find wooden spoon
uncover pot
stir
cover pot
balance wooden spoon precariously on pot handle
wait 10 minutes
goto first step (or shut off burner when done)
\end{verbatim}
Do not bump wooden spoon or it will fall.


\end{enumerate}

Notice again how text always lines up on 4-space indents (including
that last line which continues item 3 above). Here's a link to \href{http://foo.bar}{a
website}. Here's a link to a \href{local-doc.html}{local doc}. 
Inline math equations go in like so: $\omega = d\phi / dt$. Display
math should get its own line and be put in in double-dollarsigns:
$$I = \int \rho R^{2} dV$$
Done.
